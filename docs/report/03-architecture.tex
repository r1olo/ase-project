\FloatBarrier\section{Architecture}\label{sec:architecture}

\vskip 0.5cm
\includegraphics[width=\textwidth]{architecture_report.png}
\vskip 0.5cm

The distributed system architecture is based on a microservices-oriented decoupled pattern, ensuring modularity and independent scalability of the core game logic.
The following table details the functional responsibilities for each service.
\vskip 0.25cm
\noindent
\begin{tabularx}{\textwidth}{l X}
    \toprule
    \textbf{Service} & \textbf{Description} \\
    \midrule
    \textbf{API Gateway} & Serves as the single entry point for all client requests, preventing direct client access to internal services. It routes external requests to the appropriate microservices based on the requested endpoint. \\
    \textbf{Authentication} & Handles user-related functionalities, provides centralized identity management and secure session persistence via stateless JWT issuance. \\
    \addlinespace
    \textbf{Players} & Handles players-related functionalities, such as profile management and friendship features. \\
    \addlinespace
    \textbf{Matchmaking} & Manages the player queue in order pair users for game sessions. \\
    \addlinespace
    \textbf{Game Engine} & Manages the core game logic, including state transitions and scoring. \\
    \addlinespace
    \textbf{Card Catalogue} & Provides an interface for retrieving card data. \\
    \bottomrule
\end{tabularx}
\vskip 0.25cm
All of the microservices described above are implemented in Python.
Nginx is the only third-party sofware used as the API Gateway.

\subsection{Design Choices}
The architectural design splits data and functionalities among microservices, ensuring that each component serves as the authoritative owner of the data most relevant to its specific domain logic.
\begin{itemize}
    \item \textbf{Authentication Service} manages \textit{Users Database}. Since this service handles authentication and authorization logic, it must own the persistent user profiles to enforce security and access control centrally.
    \item \textbf{Players Service} manages \textit{Players Database}. As this service is responsible for all information related to players, it must own the persistent player profiles and friendship relations to ensure data integrity.
    \item \textbf{Card Catalogue} manages the \textit{Cards Database}. This service acts as interface to the Cards Database and serves as the authoritative source for card data for the entire system, ensuring consistency.
    \item \textbf{Matchmaking Service} manages the \textit{Queue Database}. This service owns the temporary data regarding the players currently in the queue. This state is highly volatile and only relevant until a match is successfully formed.
    \item \textbf{Game Engine} manages the \textit{Matches Database} and \textit{Rounds Database}. This service manages the persistent data related to active and completed matches, as well as individual rounds within those matches, since it is responsible for the core game logic and state transitions.
\end{itemize}

We decided to split the database used by Authentication Service and Players Service into a Users Database and a Players Database in order to protect users' sensitive data.
In fact, the first one contains user credentials, while the second one does not contain any private or confidential data. 

\subsection{Inter-Service Communication}
The key service-to-service dependencies are justified as follows.
\begin{itemize}
    \item \textbf{Matchmaking} to \textbf{Players}: Matchmaking Service is connected to Players Service in order to ensure data integrity, by validating that the requesting user possesses a player profile. This prevents unauthorized users from entering the queue and start a game.
    \item \textbf{Matchmaking} to \textbf{Game Engine}: Matchmaking Service is connected to Game Engine Service in order to orchestrate the transition from the queue to an active game session by triggering a match initialization, after pairing two enqueued players.
    \item \textbf{Game Engine} to \textbf{Players}: Game Engine Service is connected to Players Service in order to validate permissions. This ensures match history is only shared between players with a friendship relation, thereby enforcing data privacy.
    \item \textbf{Game Engine} to \textbf{Card Catalogue}: Game Engine Service is connected to Card Catalogue in order to validate deck submissions against the authoritative card registry. This prevents invalid or tampered cards from being used in matches, assuring fairness and security.
\end{itemize}
