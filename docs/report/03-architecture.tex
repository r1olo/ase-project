\FloatBarrier\section{Architecture}\label{sec:architecture}

\vskip 0.5cm
\includegraphics[width=\textwidth]{architecture_report.png}
\vskip 0.5cm

The distributed system architecture is based on a microservices-oriented decoupled pattern, ensuring modularity and independent scalability of the core game logic. The following table details the functional responsibilities for each service.
\vskip 0.25cm
\noindent
\begin{tabularx}{\textwidth}{l X}
    \toprule
    \textbf{Service} & \textbf{Description} \\
    \midrule
    \textbf{API Gateway} & Serves as the single entry point for all client requests, preventing direct client access to internal services. It routes external requests to the appropriate microservices based on the requested endpoint. \\
    \textbf{Authentication} & Handles user-related functionalities, provides centralized identity management and secure session persistence via stateless JWT issuance. \\
    \addlinespace
    \textbf{Players} & Handles players-related functionalities, such as profile management and friendship features. \\
    \addlinespace
    \textbf{Matchmaking} & Manages the player queue in order pair users for game sessions. \\
    \addlinespace
    \textbf{Game Engine} & Manages the core game logic, including state transitions and scoring. \\
    \addlinespace
    \textbf{Card Catalogue} & Provides an interface for retrieving card data. \\
    \bottomrule
\end{tabularx}
\vskip 0.25cm

\subsection{Design Choices}
The architectural design splits data and functionalities among microservices, ensuring that each component serves as the authoritative owner of the data most relevant to its specific domain logic.
\begin{description}
    \item[Authentication Service] manages \textit{Users Database}.
    
    \item[Players Service] manages \textit{Players Database} Database. Since this service handles authentication and friendship logic, it must own the persistent user profiles and relationship metadata to enforce security and access control centrally.
    
    \item \textbf{Card Catalogue} maintains the \textit{Static Game Definitions}. As the data describing cards and regions is reference-heavy and rarely changes, it is decoupled into this service to act as a read-optimized Single Source of Truth for the entire system.
    
    \item \textbf{Matchmaking Service} controls the \textit{Ephemeral Queue State}. This service owns the temporary data regarding active tickets and waiting players because this state is highly volatile and only relevant until a match is successfully formed.
    
    \item \textbf{Game Engine} handles the \textit{Real-Time Match State}. It manages the dynamic data of ongoing games (e.g., turns, scores, board status) in memory, as this data requires high-frequency updates and low-latency access that would be inefficient to store in a standard persistent database.
\end{description}

\subsection{Inter-Service Communication}
The key service-to-service dependencies are justified as follows.\begin{itemize}
    \item \textbf{API Gateway} to Microservices: API Gateway acts as the entry point for the system, handling request routing to all internal microservices, load balancing, and enforcing security policies.
    \item \textbf{Matchmaking} to \textbf{Players}: Matchmaking Service is connected to Players Service in order to ensure data integrity, by validating that the requesting user possesses a player profile. This prevents unauthorized users from entering the queue and start a game.
    \item \textbf{Matchmaking} to \textbf{Game Engine}: Matchmaking Service is connected to Game Engine Service in order to orchestrate the transition from the queue to an active game session by triggering a match initialization, after pairing two enqueued players.
    \item \textbf{Game Engine} to \textbf{Players}: Game Engine Service is connected to Players Service in order to validate permissions. This ensures match history is only shared between players with a friendship relation, thereby enforcing data privacy.
    \item \textbf{Game Engine} to \textbf{Card Catalogue}: Game Engine Service is connected to Card Catalogue in order to validate deck submissions against the authoritative card registry. This prevents invalid or tampered cards from being used in matches, assuring fairness and security.
\end{itemize}
