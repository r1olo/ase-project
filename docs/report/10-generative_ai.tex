\FloatBarrier\section{Use of Generative AI}\label{sec:generative-ai}

\subsection{Which AI systems did you use?}
We utilized various AI systems, such as \textbf{Gemini Pro} (provided free of charge by Google for students) and \textbf{ChatGPT}. We did not use GitHub Copilot's agent features as we did not have access to them through the university.

\subsection{What was the purpose of your project, and how did AI support it?}
The project's objective was to foster teamwork, tackle larger-scale software challenges, and acquire practical, modern skills. We experienced the entire project lifecycle, from scratch to implementation, adopting a productive \textit{DevOps} mindset. 
A key example is the use of \textbf{User Stories}; while they might seem simple (gamified), they are crucial for defining software requirements before coding, ultimately saving time on architectural decisions.

\subsection{What advantages did you notice using AI?}
The main advantage of using AI is that it accelerates understanding and coding, especially when dealing with unfamiliar languages—as was the case with \textbf{Python}. 
While it is essential to review the output for errors, well-crafted prompts yield correct solutions approximately 90\% of the time. AI models are particularly well-trained for software engineering contexts, making them highly effective tools.

\subsection{What disadvantages or limitations did you encounter?}
A potential disadvantage is \textbf{over-reliance}; since AI is fast and easy to use, it may inadvertently discourage deep critical thinking. 
Regarding limitations, the technology is improving rapidly, so we did not encounter major blockers. However, a persistent issue is \textit{hallucination}. Ideally, the AI should act more like a strict supervisor rather than an overly accommodating assistant, prioritizing accuracy over simply providing an answer.