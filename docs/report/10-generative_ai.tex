\FloatBarrier\section{Use of Generative AI}\label{sec:generative-ai}

We utilized various AI systems, such as \textbf{Gemini Pro} (provided free of charge by Google for students) and \textbf{ChatGPT}.\footnote{We did not use GitHub Copilot's agent features as we did not have access to them through the university.}

The main advantage of using AI is that it accelerates understanding and coding, especially when dealing with unfamiliar languages or emerging technologies, or complex new methodologies.
While it is essential to review the output for errors, well-crafted prompts yield correct solutions most of the time.
AI models are particularly well-trained for software engineering contexts, making them highly effective tools.

On the other hand, potential disadvantage is \textit{over-reliance}. In fact, since AI is fast and easy to use, it may inadvertently discourage deep critical thinking. 
Regarding limitations, the technology is improving rapidly, so we did not encounter major blockers.
However, a persistent issue is \textit{hallucination}.
Ideally, the AI should act more like a strict supervisor rather than an overly accommodating assistant, prioritizing accuracy over simply providing an answer.
