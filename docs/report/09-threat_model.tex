\FloatBarrier\section{Threat Model}\label{sec:threat-model}

\subsection{Work from a Model}
\includegraphics[width=\textwidth]{threat_model.png}

\FloatBarrier\subsection{Identify Assets}
\begin{itemize}
    \item \textbf{A1: Identity \& Session Data} \\
    Credentials (password hash), session identifiers.
    \item \textbf{A2: Configuration Secrets} \\
    Service certificates and keys, JWT signing keys.
    \item \textbf{A3: Game State \& Logic} \\
    Data for playing: cards, matches, rounds, queue.
    \item \textbf{A4: User PII (Personally Identifiable Information)} \\
    Stored user data such as usernames and regions, and potential future sensitive data (e.g.\@ birthdates) subject to GDPR regulation.
    \item \textbf{A5: Infrastructure \& Availability} \\
    Docker containers.
    \item \textbf{A6: Logs} \\
    Nginx logs.
    \item \textbf{A7: Ranking \& Reputation} \\
    Integrity of player scores, match history/results, leaderboards, and reputation systems (protection against tampering or cheating).
\end{itemize}

\subsection{Identify Attack Surfaces}
\begin{itemize}
    \item \textbf{Network Attack Surface}: Consists of the API Gateway (Nginx), which serves as the entry point for all external traffic.
    \item \textbf{Application Attack Surface}: Composed of the public REST API endpoints defined in the OpenAPI specification.
    \item \textbf{Internal Attack Surface}: Represented by the internal Docker network, inter-service communication, and access to persistent datastores protected by credentials injected via Docker secrets.
    \item \textbf{Supply Chain Attack Surface}: Includes external Python dependencies and base Docker images. Vulnerabilities in these third-party components could be exploited to compromise the application runtime.
\end{itemize}

\paragraph{External Attack Surfaces}
\begin{itemize}
    \item \api{/login}
    \item \api{/register}
\end{itemize}

\paragraph{Internal Attack Surfaces}
\begin{itemize}
    \item \api{/internal/players/validation}
\end{itemize}

\subsection{Identify Trust Boundaries}
\begin{itemize}
    \item \textbf{TB1: External Network Boundary (Internet vs. API Gateway)}
    \item \textbf{TB2: Application Boundary (Microservices)} 
    \item \textbf{TB3: Persistence Boundary (Microservices vs. Databases)}
\end{itemize}

\includegraphics[width=0.95\textwidth]{trust_boundaries.png}

\FloatBarrier\subsection{Identify Threats (STRIDE)}
\paragraph{S - Spoofing}
\begin{itemize}
    \item \textbf{T1: JWT Token Impersonation (Spoofing Identity)} \\
    An attacker steals a valid access token (via XSS) gaining access to authenticated endpoints like \api{/players/me} as the victim (Affects \textbf{A1}; Crosses \textbf{TB1}).
\end{itemize}

\paragraph{T - Tampering}
\begin{itemize}
    \item \textbf{T2: Injection Attacks (SQL/NoSQL)} \\
    Malicious payloads injected into input fields (e.g., username in \api{/players/search}) manipulate the backend database queries, potentially corrupting data (Affects \textbf{A3}, \textbf{A4}; Crosses \textbf{TB3}).
\end{itemize}

\paragraph{R - Repudiation}
\begin{itemize}
    \item \textbf{T3: Lack of Proof for Game Actions} \\
    If the Game Engine does not log every move with a timestamp and user signature, a player could deny having made a losing move or claim the server glitched, and there would be no audit trail to verify the claim (Affects \textbf{A7}).
\end{itemize}

\paragraph{I - Information Disclosure}
\begin{itemize}
    \item \textbf{T4: Verbose Error Leakage} \\
    The application returns stack traces or database errors (e.g., ``IntegrityError``) in the HTTP response body when an exception occurs, revealing internal structure or logic to an attacker (Affects \textbf{A5}; Crosses \textbf{TB1}).
\end{itemize}

\paragraph{D - Denial of Service}
\begin{itemize}
    \item \textbf{T5: Matchmaking Queue Saturation} \\
    An attacker uses a botnet to flood the \api{POST /enqueue} endpoint with valid tokens, filling the Redis queue and preventing legitimate players from finding a match (Affects \textbf{A3}, \textbf{A5}; Crosses \textbf{TB2}).
\end{itemize}

\paragraph{E - Elevation of Privilege}
\begin{itemize}
    \item \textbf{T6: Bypassing Gateway to Access Internal APIs} \\
    If the internal network ports (e.g., 5000) are accidentally exposed or if a container is compromised, an attacker could call \api{/internal/players/validation} directly, bypassing the authentication checks enforced by the Gateway (Affects \textbf{A1}, \textbf{A5}; Crosses \textbf{TB2}).
\end{itemize}

\subsection{Mitigate Threats}
\begin{itemize}
    \item \textbf{M1: Cryptographic Session Verification } \\
    All authenticated endpoints require a valid RS256-signed JWT. The API Gateway and microservices enforce signature verification using the public key mounted via Docker Secrets, preventing token forgery or tampering.
    \item \textbf{M2: Short-lived Access Tokens (Addresses T1)} \\
    To minimize the window of opportunity for an attacker using a stolen token, the system issues JWTs with a short expiration time.
    \item \textbf{M3: Strict Input \& Logic Validation (Addresses T2, T3, T5)} \\
    Input data is strictly validated against the OpenAPI schema (types, formats, and patterns) before processing. Furthermore, the Game Engine for example, implements logic to verify if a player is already enqued.
    \item \textbf{M4: Network Isolation \& Access Control (Addresses T6)} \\
    Internal microservices and databases are isolated within a private Docker network, inaccessible from the public internet. Direct access is blocked by the API Gateway, which acts as the sole entry point.
    \item \textbf{M5: Generic Errors (Addresses T4)} \\
    The application is configured to return generic error messages to clients, suppressing stack traces to prevent information leakage.
\end{itemize}
