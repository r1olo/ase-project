\section{Threat model}\label{sec:threat-model}
\subsection{Work from a Model}

\includegraphics[width=\textwidth]{model.drawio.png}

\subsection{Identify Assets}

Di seguito sono elencati gli asset critici identificati nel sistema:

\begin{itemize}
    \item \textbf{A1: Identity \& Session Data} \\
    Credentials (password hash), session identifiers, refresh tokens, access tokens.
    
    \item \textbf{A2: Configuration Secrets} \\
    Application secrets, JWT signing keys , database connection strings, API keys.
    
    \item \textbf{A3: Game State \& Logic} \\
    Data for playing: cards, matches, rounds , queue.

    \item \textbf{A4: User PII (Personally Identifiable Information)} \\
    Stored user data such as usernames and regions, and potential future sensitive data (e.g. birthdates) subject to GDPR regulation.

    \item \textbf{A5: Infrastructure \& Availability} \\
    Docker containers, system resources (CPU, RAM), network bandwidth, and overall service uptime (availability against DoS attacks).

    \item \textbf{A6: Logs \& Audit Trails} \\
    Application logs generated by Flask, access logs, debug traces, and security audit records necessary for forensics.

    \item \textbf{A7: Ranking \& Reputation} \\
    Integrity of player scores, match history/results, leaderboards, and reputation systems (protection against tampering or cheating).
\end{itemize}
\subsection{Identify Attack Surfaces}

The possible attack surfaces identified in the architecture are:

\begin{itemize}
    \item \textbf{Network Attack Surface}: Consists of the API Gateway (Nginx), which handles TLS termination and serves as the entry point for all external traffic.
    
    \item \textbf{Application Attack Surface}: Composed of the public REST API endpoints defined in the OpenAPI specification.
    
    \item \textbf{Internal Attack Surface}: Represented by the internal Docker network, inter-service communication, and access to persistent datastores protected by credentials injected via Docker secrets.

    \item \textbf{Supply Chain Attack Surface}: Includes external Python dependencies and base Docker images. Vulnerabilities in these third-party components could be exploited to compromise the application runtime.
\end{itemize}

\subsubsection*{External Attack Surfaces}
The external surface is primarily defined by the \textbf{API Gateway (Nginx)} exposed on port 443, which acts as the single entry point for all traffic, handling TLS termination and routing to upstream services. It exposes the \textbf{Public API Endpoints}, including unauthenticated routes (\texttt{/login}, \texttt{/register}) accessible to anonymous actors, and authenticated routes (\texttt{/players}, \texttt{/matches}) which, despite requiring valid JWTs, remain susceptible to attacks from malicious registered users.

\subsubsection*{Internal Attack Surfaces}
The internal surface encompasses privileged components accessible only within the trusted network, such as \textbf{Internal Service APIs} (e.g., \texttt{/internal/players/validation}) designed for inter-service communication. It includes \textbf{Backend Data Services} like PostgreSQL databases and Redis caches protected by Docker Secrets, and the \textbf{Infrastructure \& Deployment} layer (CI/CD pipelines), where access allows modification of source code and runtime configurations.